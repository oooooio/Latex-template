%% 
%% Copyright 2007-2025 Elsevier Ltd
%% 
%% This file is part of the 'Elsarticle Bundle'.
%% ---------------------------------------------
%% 
%% It may be distributed under the conditions of the LaTeX Project Public
%% License, either version 1.3 of this license or (at your option) any
%% later version.  The latest version of this license is in
%%    http://www.latex-project.org/lppl.txt
%% and version 1.3 or later is part of all distributions of LaTeX
%% version 1999/12/01 or later.
%% 
%% The list of all files belonging to the 'Elsarticle Bundle' is
%% given in the file `manifest.txt'.
%% 
%% Template article for Elsevier's document class `elsarticle'
%% with harvard style bibliographic references

\documentclass[preprint,12pt,authoryear]{elsarticle}

%% Use the option review to obtain double line spacing
%% \documentclass[authoryear,preprint,review,12pt]{elsarticle}

%% Use the options 1p,twocolumn; 3p; 3p,twocolumn; 5p; or 5p,twocolumn
%% for a journal layout:
%% \documentclass[final,1p,times,authoryear]{elsarticle}
%% \documentclass[final,1p,times,twocolumn,authoryear]{elsarticle}
%% \documentclass[final,3p,times,authoryear]{elsarticle}
%% \documentclass[final,3p,times,twocolumn,authoryear]{elsarticle}
%% \documentclass[final,5p,times,authoryear]{elsarticle}
%% \documentclass[final,5p,times,twocolumn,authoryear]{elsarticle}

%% For including figures, graphicx.sty has been loaded in
%% elsarticle.cls. If you prefer to use the old commands
%% please give \usepackage{epsfig}

\input{common-info.tex} %% Include common parameters file
%% The amssymb package provides various useful mathematical symbols
\usepackage{amssymb}
%% The amsmath package provides various useful equation environments.
\usepackage{amsmath}
%% The amsthm package provides extended theorem environments
%% \usepackage{amsthm}

%% The lineno packages adds line numbers. Start line numbering with
%% \begin{linenumbers}, end it with \end{linenumbers}. Or switch it on
%% for the whole article with \linenumbers.
%% \usepackage{lineno}

\journal{\journalName}

\begin{document}

\begin{frontmatter}

  \title{\myTitle} %% Article title


  \author[1]{\myName} %% Author name
  \author[]{\correspondingAuthorName \corref{cor1}}
  \cortext[cor1]{Corresponding author}

  %% Author affiliation
  \affiliation[1]{
    organization={\institutionOrg},
    addressline={\institutionAddress},
    city={\institutionCity},
    postcode={\institutionPostcode},
    state={\institutionState},
    country={\institutionCountry}
  }

  %% Abstract
  \begin{abstract}
    %% Text of abstract
    Abstract text.
  \end{abstract}

  %%Graphical abstract
  \begin{graphicalabstract}
    %\includegraphics{grabs} 
  \end{graphicalabstract}
 
  %%Research highlights
  \highlight

  %% Keywords
  \begin{keyword}
    %% keywords here, in the form: keyword \sep keyword

    %% PACS codes here, in the form: \PACS code \sep code

    %% MSC codes here, in the form: \MSC code \sep code
    %% or \MSC[2008] code \sep code (2000 is the default)

  \end{keyword}

\end{frontmatter}

%% Add \usepackage{lineno} before \begin{document} and uncomment 
%% following line to enable line numbers
%% \linenumbers

%% main text
%%

\section{Introduction}
\label{sec:introduction}


\section{Datasets and Methods}
\label{sec:data-methods}

\section{Results}
\label{sec:results}

\section{Discussion}
\label{sec:discussion}

\section{Conclusion}
\label{sec:conclusion}




%% The Appendices part is started with the command \appendix;
%% appendix sections are then done as normal sections
\appendix
\section{Example Appendix Section}
\label{app1}

Appendix text.

\bibliographystyle{elsarticle-harv}
\bibliography{references.bib}

\end{document}

\endinput
%%
%% End of file `elsarticle-template-harv.tex'.


